% Plantilla para documentos LaTeX para enunciados
% Por Pedro Pablo Aste Kompen - ppaste@uc.cl
% Licencia Creative Commons BY-NC-SA 3.0
% http://creativecommons.org/licenses/by-nc-sa/3.0/

\documentclass[12pt]{article}

% Margen de 1 pulgada por lado
\usepackage{fullpage}
% Incluye gráficas
\usepackage{graphicx}
% Packages para matemáticas, por la American Mathematical Society
\usepackage{amssymb}
\usepackage{amsmath}
% Desactivar hyphenation
\usepackage[none]{hyphenat}
% Saltar entre párrafos - sin sangrías
\usepackage{parskip}
% Español y UTF-8
\usepackage[spanish]{babel}
\usepackage[utf8]{inputenc}
% Links en el documento
\usepackage{hyperref}
\usepackage{fancyhdr}
\usepackage{bookmark}
\usepackage{caption}
\setlength{\headheight}{15.2pt}
\setlength{\headsep}{5pt}
\pagestyle{fancy}

\newcommand{\N}{\mathbb{N}}
\newcommand{\Exp}[1]{\mathcal{E}_{#1}}
\newcommand{\List}[1]{\mathcal{L}_{#1}}
\newcommand{\EN}{\Exp{\N}}
\newcommand{\LN}{\List{\N}}

\newcommand{\comment}[1]{}
\newcommand{\lb}{\\~\\}
\newcommand{\eop}{_{\square}}
\newcommand{\hsig}{\hat{\sigma}}
\newcommand{\ra}{\rightarrow}
\newcommand{\lra}{\leftrightarrow}

% Cambiar por nombre completo + número de alumno
\newcommand{\alumno}{Paul Heinsohn Manetti - 1562305J}
\rhead{Informe Tarea 3 - \alumno}

\begin{document}
\thispagestyle{empty}
% Membrete
% PUC-ING-DCC-IIC1103
\begin{minipage}{2.3cm}
\includegraphics[width=2cm]{img/logo.pdf}
\vspace{0.5cm} % Altura de la corona del logo, así el texto queda alineado verticalmente con el círculo del logo.
\end{minipage}
\begin{minipage}{\linewidth}
\textsc{\raggedright \footnotesize
Pontificia Universidad Católica de Chile \\
Departamento de Ciencia de la Computación \\
IIC2133 - Estructuras de Datos y Algoritmos \\}
\end{minipage}

% Titulo
\begin{center}
\vspace{0.5cm}
{\huge\bf Informe Tarea 3}\\
\vspace{0.2cm}
\today\\
\vspace{0.2cm}
\footnotesize{1º semestre 2019 - Profesor Yadran Eterovic}\\
\vspace{0.2cm}
\footnotesize{\alumno}
\rule{\textwidth}{0.05mm}
\end{center}

\section*{Introducción}
\hspace{5mm}
En primer lugar, se mencionará una poda y una heurística para posteriormente realizar un análisis de estas en base 
a su influencia en el desempeño del algoritmo que busca soluciones para ZHEDD. Sin embargo, antes quisiera destacar
que no se mencionarán dos de cada una debido a que no alcancé a implementarlas en mi código para la fecha de entrega.


\section*{Resultado \textit{Naive}}
\hspace{5mm}
En la sección de 'Anexos', en las tablas del 1 al 4, se muestran la cantidad de \textit{undo's} y tiempo de ejecución para
todos los tests (1 al 30) mediante la ejecución del código \textit{naive} de \textit{backtracking}, es decir, un DFS con 
detección de ciclos cuya complejidad es factorial en base a la cantidad de puentes.

\hspace{5mm}
A partir de estos resultados, es posible apreciar que desde el \textit{test} 13 en adelante los tiempos de ejecución
comienzan a sobrepasar los 30 segundos a excepción del 16, es más, a partir del 22 no se encuentran soluciones en 
menos de 4 minutos, por lo que es posible afirmar que nuestra solución \textit{naive} es ineficiente para este problema.

\newpage

\section*{Heurística}
\hspace{5mm}
La heurística aplicada consiste en la obtención de aquellos puentes que están alineados con el objetivo, debido a que 
uno de estos contendrá un movimiento, dado el tablero, que conduzca a la solución del problema. Es por lo anterior que se
fija un arreglo de puentes en donde los potenciales ganadores son posicionados al final de dicho arreglo, así estos serán
los últimos en realizar un movimiento al momento de probar combinaciones de estados. Lo anterior aplica como heurística ya
que propone una estrategia que conduce a la resolución del problema antes de que se verifiquen algunos estados cuyo último
movimiento sean realizados por puentes que no pueden llegar al objetivo, por lo que reduce la cantidad de \textit{undo's}
del \textit{backtracking}. Estos resultados pueden encontrarse en la sección de 'Anexos' desde la tabla 5 a la 8.

\hspace{5mm}
A partir de lo anterior, es posible apreciar que para los primeros 14 \textit{tests} no hay muchos cambios, es más, 
en algunos casos el tiempo disminuye mientras que incrementan la cantidad de \textit{undo's}, por ejemplo los
\textit{tests} 6 y 9, por lo que la heurística para estos casos no fue influyente. Por otra parte, si bien los
\textit{tests} 16 y 17 disminuyen tanto la cantidad de \textit{undo's} como el tiempo de ejecución en un par de 
segundos, y el \textit{test} 21 pasa de $\infty$ a 66 segundos, el \textit{test} 15 y particularmente el 20
incrementan de forma notoria. Es por lo anterior que se puede concluir que dicha heurística, por sí sola, no realiza
grandes aportes a la resolución de problemas de este tipo, es más, puede favorecer a unos y perjudicar a otros.


\section*{Poda}
\hspace{5mm}
La poda aplicada consiste en dejar de expandir ramas cuando un puente que está presente en alguno de los bordes 
del cuadrante formado por los ejes (\textit{x}, \textit{y}) de los puentes apunta fuera de éste. Esta poda es útil
ya que permite dejar de expandir nodos con puentes cuya dirección no aporta al acortamiento de la distancia
Manhattan entre la unión de los puentes con el objetivo. Esta poda, además, evita que el ganador apunte en la dirección
contraria al objetivo, acortando aun más el espacio de búsqueda. Los resultados al aplicarle esta poda al
\textit{backtracking naive} son los siguientes: Ver sección 'Anexos', tablas del 9 al 12.

\hspace{5mm}
A partir de lo anterior, es posible apreciar que en todos los \textit{tests} distintos de $\infty$ se presentan
cambios notorios en la cantidad de \textit{undo's}, tanto es así que algunos \textit{tests} como el 13 y 15 disminuyen
su tiempo de ejecución de forma notoria. Sin embargo, esta poda por sí sola no es lo suficientemente poderosa
para poder llevar a aquellos \textit{tests} que se demoran más de 4 minutos bajo dicha cifra.


\section*{Heurística y poda}
\hspace{5mm}
Finalmente, se experimentó con la inclusión de ambos métodos mencionados (heurística más poda), y se obtuvieron 
los resultados mostrados en las tablas de la 13 a la 16. A partir de estos, es posible apreciar que se parecen bastante a los
obtenidos por la poda sin heurística. Es por lo anterior que se concluye que la heurística implementada no aporta a este tipo
de problema podados, es más, quizás no aporta con ningún otro tipo de poda, pero para afirmar esto habría que realizar más
experimentos con distintas podas, e incluso con otras heurísticas afirmar su influencia en este tipo de problemas.
Por otra parte, se destaca que la poda implementada, a pesar de ser una poda bastante simple y fácil de implementar, logró disminuir
notoriamente tanto la cantidad de \textit{undo's} como los tiempos de ejecución para la mayoría de los \textit{tests}, por lo 
que definitivamente la puedo considerar una poda importante que debe ser tomada en consideración al momento de implementar un
algoritmo eficiente de \textit{backtracking} para problemas de este tipo.

\section*{Anexos}

\textit{* $\infty$: tiempo mayor a 4 minutos, y por ende la cantidad de undo's es demasiado grande} \\

\begin{center}
    \begin{table}[h]
        \caption{"\textit{Naive} - Tiempos de ejecución (en segundos) vs \textit{undo's} de los tests 01 al 09"}
        \centering
        \begin{tabular} {| c | c | c | c | c | c | c | c | c | c |}
            \hline
            Test & T01 & T02 & T03 & T04 & T05 & T06 & T07 & T08 & T09 \\
            \hline
            Tiempo & 0.018 & 0.019 & 0.019 & 0.019 & 0.018 & 0.019 & 0.021 & 0.098 & 0.029 \\
            \textit{Undo's} & 2 & 3 & 12 & 10 & 2076 & 2055 & 23668 & 1161889 & 90052 \\
            \hline
        \end{tabular}
    \end{table}
    \begin{table}[h]
        \caption{"\textit{Naive} - Tiempos de ejecución (en segundos) vs \textit{undo's} de los tests 10 al 16"}
        \centering
        \begin{tabular} {| c | c | c | c | c | c | c | c | c |}
            \hline
            Test & T10 & T11 & T12 & T13 & T14 & T15 & T16 \\
            \hline
            Tiempo & 0.022 & 0.053 & 0.068 & 59.538 & $\infty$ & 30.975 & 7.140 \\
            \textit{Undo's} & 41188 & 335115 & 522748 & 656154787 & $\infty$ & 342618420 & 91328418 \\
            \hline
        \end{tabular}
    \end{table}
    \begin{table}[h]
        \caption{"\textit{Naive} - Tiempos de ejecución (en segundos) vs \textit{undo's} de los tests 17 al 22"}
        \centering
        \begin{tabular} {| c | c | c | c | c | c | c | c |}
            \hline
            Variable & T17 & T18 & T19 & T20 & T21 & T22 \\
            \hline
            Tiempo & 4.803 & 88.035 & 181.777 & 40.088 & 60.610 & $\infty$ \\
            \textit{Undo's} & 43845763 & 981581294 & 1846291824 & 541580316 & 664125782 & $\infty$ \\
            \hline
        \end{tabular}
    \end{table}
    \begin{table}[h]
        \caption{"\textit{Naive} - Tiempos de ejecución (en segundos) vs \textit{undo's} de los tests 23 al 30"}
        \centering
        \begin{tabular} {| c | c | c | c | c | c | c | c | c | c | c | c | c | c | c |}
            \hline
            Variable & T23 & T24 & T25 & T26 & T27 & T28 & T29 & T30 \\
            \hline
            Tiempo & $\infty$ & $\infty$ & $\infty$ & $\infty$ & $\infty$ & $\infty$ & $\infty$ & $\infty$ \\
            \textit{Undo's} & $\infty$ & $\infty$ & $\infty$ & $\infty$ & $\infty$ & $\infty$ & $\infty$ & $\infty$ \\
            \hline
        \end{tabular}
    \end{table}
    \begin{table}[h]
        \caption{"Heurística - Tiempos de ejecución (en segundos) vs \textit{undo's} de los tests 01 al 10"}
        \centering
        \begin{tabular} {| c | c | c | c | c | c | c | c | c | c |}
            \hline
            Test & T01 & T02 & T03 & T04 & T05 & T06 & T07 & T08 & T09 \\
            \hline
            Tiempo & 0.020 & 0.016 & 0.017 & 0.014 & 0.015 & 0.015 & 0.018 & 0.094 & 0.022 \\
            \textit{Undo's} & 2 & 3 & 12 & 10 & 2076 & 2239 & 23668 & 1162033 & 90196 \\
            \hline
        \end{tabular}
    \end{table}
    \begin{table}[h]
        \caption{"Heurística - Tiempos de ejecución (en segundos) vs \textit{undo's} de los tests 10 al 16"}
        \centering
        \begin{tabular} {| c | c | c | c | c | c | c | c | c |}
            \hline
            Test & T10 & T11 & T12 & T13 & T14 & T15 & T16 \\
            \hline
            Tiempo & 0.021 & 0.052 & 0.061 & 57.051 & $\infty$ & 96.125 & 1.109 \\
            \textit{Undo's} & 41080 & 332979 & 522748 & 656184371 & $\infty$ & 1119926400 & 15559826 \\
            \hline
        \end{tabular}
    \end{table}
    \begin{table}[h]
        \caption{"Heurística - Tiempos de ejecución (en segundos) vs \textit{undo's} de los tests 17 al 22"}
        \centering
        \begin{tabular} {| c | c | c | c | c | c | c | c |}
            \hline
            Variable & T17 & T18 & T19 & T20 & T21 & T22 \\
            \hline
            Tiempo & 2.059 & 89.026 & 184.335 & $\infty$ & 61.122 & $\infty$ \\
            \textit{Undo's} & 20115331 & 1846323196 & 1846291824 & $\infty$ & 664125782 & $\infty$ \\
            \hline
        \end{tabular}
    \end{table}
    \begin{table}[h]
        \caption{"Heurística - Tiempos de ejecución (en segundos) vs \textit{undo's} de los tests 23 al 30"}
        \centering
        \begin{tabular} {| c | c | c | c | c | c | c | c | c | c | c | c | c | c | c |}
            \hline
            Variable & T23 & T24 & T25 & T26 & T27 & T28 & T29 & T30 \\
            \hline
            Tiempo & $\infty$ & $\infty$ & $\infty$ & $\infty$ & $\infty$ & $\infty$ & $\infty$ & $\infty$ \\
            \textit{Undo's} & $\infty$ & $\infty$ & $\infty$ & $\infty$ & $\infty$ & $\infty$ & $\infty$ & $\infty$ \\
            \hline
        \end{tabular}
    \end{table}
    \begin{table}[h]
        \caption{"Poda - Tiempos de ejecución (en segundos) vs \textit{undo's} de los tests 01 al 10"}
        \centering
        \begin{tabular} {| c | c | c | c | c | c | c | c | c | c |}
            \hline
            Test & T01 & T02 & T03 & T04 & T05 & T06 & T07 & T08 & T09 \\
            \hline
            Tiempo & 0.020 & 0.016 & 0.015 & 0.015 & 0.067 & 0.017 & 0.017 & 0.047 & 0.021 \\
            \textit{Undo's} & 2 & 3 & 7 & 4 & 929 & 1237 & 7085 & 374209 & 39532 \\
            \hline
        \end{tabular}
    \end{table}
    \begin{table}[h]
        \caption{"Poda - Tiempos de ejecución (en segundos) vs \textit{undo's} de los tests 10 al 16"}
        \centering
        \begin{tabular} {| c | c | c | c | c | c | c | c | c |}
            \hline
            Test & T10 & T11 & T12 & T13 & T14 & T15 & T16 \\
            \hline
            Tiempo & 0.017 & 0.022 & 0.030 & 12.205 & $\infty$ & 20.413 & 2.994 \\
            \textit{Undo's} & 23692 & 44685 & 121795 & 111814507 & $\infty$ & 196233133 & 30351649 \\
            \hline
        \end{tabular}
    \end{table}
    \begin{table}[h]
        \caption{"Poda - Tiempos de ejecución (en segundos) vs \textit{undo's} de los tests 17 al 22"}
        \centering
        \begin{tabular} {| c | c | c | c | c | c | c | c |}
            \hline
            Variable & T17 & T18 & T19 & T20 & T21 & T22 \\
            \hline
            Tiempo & 1.730 & 34.099 & 57.233 & 11.959 & 22.398 & $\infty$ \\
            \textit{Undo's} & 14526204 & 335870016 & 475741045 & 145816990 & 218709456 & $\infty$ \\
            \hline
        \end{tabular}
    \end{table}
    \begin{table}[h]
        \caption{"Poda - Tiempos de ejecución (en segundos) vs \textit{undo's} de los tests 23 al 30"}
        \centering
        \begin{tabular} {| c | c | c | c | c | c | c | c | c | c | c | c | c | c | c |}
            \hline
            Variable & T23 & T24 & T25 & T26 & T27 & T28 & T29 & T30 \\
            \hline
            Tiempo & $\infty$ & $\infty$ & $\infty$ & $\infty$ & $\infty$ & $\infty$ & $\infty$ & $\infty$ \\
            \textit{Undo's} & $\infty$ & $\infty$ & $\infty$ & $\infty$ & $\infty$ & $\infty$ & $\infty$ & $\infty$ \\
            \hline
        \end{tabular}
    \end{table}
    \begin{table}[h]
        \caption{"H + Poda - Tiempos de ejecución (en segundos) vs \textit{undo's} de los tests 23 al 30"}
        \centering
        \begin{tabular} {| c | c | c | c | c | c | c | c | c | c |}
            \hline
            Test & T01 & T02 & T03 & T04 & T05 & T06 & T07 & T08 & T09 \\
            \hline
            Tiempo & 0.023 & 0.018 & 0.014 & 0.015 & 0.016 & 0.015 & 0.016 & 0.048 & 0.021 \\
            \textit{Undo's} & 1 & 2 & 7 & 4 & 929 & 1343 & 7085 & 374293 & 39287 \\
            \hline
        \end{tabular}
    \end{table}
    \begin{table}[h]
        \caption{"H + Poda - Tiempos de ejecución (en segundos) vs \textit{undo's} de los tests 23 al 30"}
        \centering
        \begin{tabular} {| c | c | c | c | c | c | c | c | c |}
            \hline
            Test & T10 & T11 & T12 & T13 & T14 & T15 & T16 \\
            \hline
            Tiempo & 0.017 & 0.022 & 0.029 & 12.303 & $\infty$ & 34.566 & 0.452 \\
            \textit{Undo's} & 23625 & 43932 & 121795 & 111824455 & $\infty$ & 496253133 & 5143753 \\
            \hline
        \end{tabular}
    \end{table}
    \begin{table}[h]
        \caption{"H + Poda - Tiempos de ejecución (en segundos) vs \textit{undo's} de los tests 23 al 30"}
        \centering
        \begin{tabular} {| c | c | c | c | c | c | c | c |}
            \hline
            Variable & T17 & T18 & T19 & T20 & T21 & T22 \\
            \hline
            Tiempo & 0.827 & 37.157 & 59.652 & 11.959 & 22.398 & $\infty$ \\
            \textit{Undo's} & 7180342 & 335869874 & 474414871 & 145816990 & 218709456 & $\infty$ \\
            \hline
        \end{tabular}
    \end{table}
    \begin{table}[h]
        \caption{"H + Poda - Tiempos de ejecución (en segundos) vs \textit{undo's} de los tests 23 al 30"}
        \centering
        \begin{tabular} {| c | c | c | c | c | c | c | c | c | c | c | c | c | c | c |}
            \hline
            Variable & T23 & T24 & T25 & T26 & T27 & T28 & T29 & T30 \\
            \hline
            Tiempo & $\infty$ & $\infty$ & $\infty$ & $\infty$ & $\infty$ & $\infty$ & $\infty$ & $\infty$ \\
            \textit{Undo's} & $\infty$ & $\infty$ & $\infty$ & $\infty$ & $\infty$ & $\infty$ & $\infty$ & $\infty$ \\
            \hline
        \end{tabular}
    \end{table}
\end{center}

\end{document}
