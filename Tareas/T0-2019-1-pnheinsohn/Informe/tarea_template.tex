% Plantilla para documentos LaTeX para enunciados
% Por Pedro Pablo Aste Kompen - ppaste@uc.cl
% Licencia Creative Commons BY-NC-SA 3.0
% http://creativecommons.org/licenses/by-nc-sa/3.0/

\documentclass[12pt]{article}

% Margen de 1 pulgada por lado
\usepackage{fullpage}
% Incluye gráficas
\usepackage{graphicx}
% Packages para matemáticas, por la American Mathematical Society
\usepackage{amssymb}
\usepackage{amsmath}
% Desactivar hyphenation
\usepackage[none]{hyphenat}
% Saltar entre párrafos - sin sangrías
\usepackage{parskip}
% Español y UTF-8
\usepackage[spanish]{babel}
\usepackage[utf8]{inputenc}
% Links en el documento
\usepackage{hyperref}
\usepackage{fancyhdr}
\usepackage{bookmark}
\setlength{\headheight}{15.2pt}
\setlength{\headsep}{5pt}
\pagestyle{fancy}

\newcommand{\N}{\mathbb{N}}
\newcommand{\Exp}[1]{\mathcal{E}_{#1}}
\newcommand{\List}[1]{\mathcal{L}_{#1}}
\newcommand{\EN}{\Exp{\N}}
\newcommand{\LN}{\List{\N}}

\newcommand{\comment}[1]{}
\newcommand{\lb}{\\~\\}
\newcommand{\eop}{_{\square}}
\newcommand{\hsig}{\hat{\sigma}}
\newcommand{\ra}{\rightarrow}
\newcommand{\lra}{\leftrightarrow}

% Cambiar por nombre completo + número de alumno
\newcommand{\alumno}{Paul Heinsohn Manetti - 1562305J}
\rhead{Informe Tarea 0 - \alumno}

\begin{document}
\thispagestyle{empty}
% Membrete
% PUC-ING-DCC-IIC1103
\begin{minipage}{2.3cm}
\includegraphics[width=2cm]{img/logo.pdf}
\vspace{0.5cm} % Altura de la corona del logo, así el texto queda alineado verticalmente con el círculo del logo.
\end{minipage}
\begin{minipage}{\linewidth}
\textsc{\raggedright \footnotesize
Pontificia Universidad Católica de Chile \\
Departamento de Ciencia de la Computación \\
IIC2133 - Estructuras de Datos y Algoritmos \\}
\end{minipage}

% Titulo
\begin{center}
\vspace{0.5cm}
{\huge\bf Informe Tarea 0}\\
\vspace{0.2cm}
\today\\
\vspace{0.2cm}
\footnotesize{1º semestre 2019 - Profesor Yadran Eterovic}\\
\vspace{0.2cm}
\footnotesize{\alumno}
\rule{\textwidth}{0.05mm}
\end{center}

\section*{Implementación}

\hspace{5mm}
Por temas de legibilidad y simplicidad de código decidí abordar la simulación mediante mi propia estructura 
de datos. Esta consiste en una variable llamada 'table' del tipo 'Node***' que funciona como una matriz de nodos, 
cuyas filas son del tipo 'Node**' y columnas del tipo 'Node*'. Gracias a lo anterior, pude acceder a estos. Cada 
nodo apunta a dos variables: 'state' y 'neighbours'. La primera es un número entero (int), mientras que la 
segunda es un puntero de punteros del mismo tipo (Node**). La idea de 'state' es almacenar el tipo de bacteria 
presente dicho nodo, mientras que 'neighbours' para guardar los nodos vecinos.

\hspace{5mm}
Esta modelación me permite realizar una iteración en base a las filas 'i' y otra en base a las columnas 'j' 
para acceder a los nodos mediante la sentencia 'table[i][j]', que a su vez facilita la asignación y el acceso 
a los vecinos para realizar los cambios pertinentes.

\section*{Complejidad}

\hspace{5mm}
En cuanto al algoritmo ejecutado en cada generación, tenemos que en su mejor caso se realizan $3 \cdot M 
\cdot N$ comparaciones, ya que basta que cada nodo sea comparado con otros tres que le ganen para cambiarlo 
de estado. En cuanto a los otros casos, es posible apreciar que realizan la misma cantidad de comparaciones 
debido a que la probabilidad de que un nodo le gane a otro es de $1 / 3$, y como se requieren tres para 
cambiarlo de estado, se tiene que en el caso promedio tres de cada ocho nodos van a ganarle al actual, y 
como $8 / 3 < 3$, necesariamente se van a realizar las ocho comparaciones al igual que en el peor caso, ergo, 
ambos realizan $8 \cdot M \cdot N$ comparaciones.

\hspace{5mm}
Es por lo anterior que el algoritmo tiene complejidad O(M$\cdot$N) en cualquier caso, lo que inicialmente uno 
tiende a pensar que no está tan mal, sin embargo, a medida de que iba probando los tests de mayor tamaño, más 
se demoraba la ejecución de cada generación. Entrando en detalle, es posible apreciar que el primer (test\_00), 
cuya matriz es de 5x5, tarda en llegar a la generación de un solo tipo de bacteria (3 en adelante) en menos de 
medio segundo, mientras que al ejecutar los últimos dos testes (test\_06 y test\_07) demoraba más de un segundo 
por generación. \\

\begin{center}
    \begin{tabular}{| c | c | c |}
        \multicolumn{3}{l}{Tabla 1: Complejidad en casos} \\
        \hline
        Caso & Comparaciones & Complejidad \\
        \hline
        Peor & $8 \cdot M \cdot N$ & $O(M \cdot N)$ \\
        Promedio & $8 \cdot M \cdot N$ & $O(M \cdot N)$ \\
        Mejor & $3 \cdot M \cdot N$ & $O(M \cdot N)$ \\
        \hline
    \end{tabular}
\end{center}
\vspace{4mm}
\hspace{5mm}
Lo anterior puede explicarse tanto por el algoritmo como por las estructuras empleadas. Seguramente, mi 
algoritmo puede ser modificado para que la cantidad de comparaciones disminuya, pero creo que lo que más 
afecta el funcionamiento de éste es el uso de estructuras, ya que mi variable 'table' podría ser solamente 
del tipo 'int**', la cual mediante el correcto uso de punteros evita la múltiple 'instanciación' de nodos, 
por lo que disminuirían los accesos a la memoria haciendo más fluido el transcurso de las generaciones. 
Finalmente, me gustaría destacar que la Implementación de 'int' en vez de 'uint\_8' también influyó en el 
desempeño del algoritmo, ya que se reservó más memoria y se terminó leyendo innecesariamente bloques que 
no aportaban información relevante.

\end{document}
